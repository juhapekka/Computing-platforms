\documentclass{article}
\usepackage[utf8]{inputenc}
\usepackage{graphicx}
\usepackage{titling}
\usepackage{titlesec}
\usepackage{booktabs}
\usepackage{fancyhdr}
\usepackage{lipsum}
\usepackage{comment}
\usepackage{enumitem}
\usepackage{listings}
\usepackage{xcolor}
\usepackage{longtable}
\usepackage{cite}


\lstdefinestyle{pidstyle}{
    basicstyle=\ttfamily\footnotesize,
    breaklines=true,
    escapechar=\#, % Define escape character for inline LaTeX commands
    linewidth=\textwidth,
    basicstyle=\ttfamily\scriptsize
}

\renewcommand{\maketitle}{%
  \begin{leftmark}
    \vspace*{\baselineskip} % Add a bit of vertical space

%    \includegraphics[width=4cm]{example-image-a} % Add an image before the title. you will need to replace the image path with your own

%    \vspace{0.5cm} % Add vertical space before title

    \textbf{\fontsize{18}{36}\selectfont \thetitle} % Font Size and Bold Title

     \vspace{0.05cm} % Add vertical space before subtitle
%    \textit{\Large \theauthor}  % Subtitle / Author
    \vspace{\baselineskip} % Add vertical space after subtitle
     \rule{\textwidth}{0.4pt} % Add a horizontal line

   \end{leftmark}
%    \thispagestyle{empty} % Prevent header/footer on the title page
}


% Section Formatting
\titleformat{\section}
  {\normalfont\fontsize{18}{22}\bfseries} % Font and style
  {\thesection}         % Section number
  {1em}                   % Horizontal space after section number
  {}                     % Code before the section name
  []                     % Code after the section name

\titleformat{\subsection}
  {\normalfont\fontsize{14}{18}\bfseries} % Font and style
  {\thesubsection}         % Subsection number
  {1em}                   % Horizontal space after subsection number
  {}                     % Code before the subsection name
  []                     % Code after the subsection name

\setlength{\parindent}{0pt}

\title{Computing platforms (Spring 2025)\newline
week 2}
\author{Juha-Pekka Heikkilä}



\pagestyle{fancy}
\fancyhf{}

\renewcommand{\headrulewidth}{0pt}

\newcommand{\footerline}{\makebox[\textwidth]{\hrulefill}}

\newcommand{\footercontent}{%
    \begin{tabular}{@{}l@{}}
        \footerline \\
        \leftmark \hfill \rlap{\thepage}
    \end{tabular}
}

\fancyfoot[C]{\footercontent}


\newcommand{\exercise}[1]{
    \section*{Exercise #1}
    \markboth{Exercise #1}{}
}



\begin{document}
\maketitle


\exercise{1}
{\bf Multicore. Suppose a single application is running on
a multicore system with 16 processors. If 20\% of the code
is inherently serial, what is the performance gain over
a single processor system? What would it be if only 2\% of
the code would be inherently serial?}

Using Amdahl's Law \cite{stallings4.3}:
\[
S = \frac{1}{(1 - f) + \frac{f}{N}}
\]
where:
\begin{itemize}
    \item \( 1 - f \): Fraction of the code that is inherently serial.
    \item \( f \): Fraction of the code that is parallelizable.
    \item \( N = 16 \): Number of processors.
\end{itemize}

\begin{enumerate}[label=\textbf{\alph*})]
    \item \textbf{20\% Serial Code (\( 1 - f = 0.2 \), \( f = 0.8 \))}

    Substitute into the formula:
    \[
    S = \frac{1}{(1 - 0.8) + \frac{0.8}{16}} = \frac{1}{0.2 + 0.05} = \frac{1}{0.25} = 4
    \]

    \textbf{Performance Gain:} \( S = 4 \).

    \item \textbf{2\% Serial Code (\( 1 - f = 0.02 \), \( f = 0.98 \))}

    Substitute into the formula:
    \[
    S = \frac{1}{(1 - 0.98) + \frac{0.98}{16}} = \frac{1}{0.02 + 0.06125} = \frac{1}{0.08125} \approx 12.31
    \]

    \textbf{Performance Gain:} \( S \approx 12.31 \).
\end{enumerate}

\newpage
\bibliographystyle{plain}
\bibliography{references}
\end{document}