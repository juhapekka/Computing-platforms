% !TEX program = xelatex
\documentclass{article}
\usepackage[utf8]{inputenc}
\usepackage{graphicx}
\usepackage{titling}
\usepackage{titlesec}
\usepackage{booktabs}
\usepackage{fancyhdr}
\usepackage{lipsum}
\usepackage{comment}
\usepackage{enumitem}
\usepackage{listings}
\usepackage{xcolor}
\usepackage{longtable}
%\usepackage{cite}
\usepackage{pgfgantt}
\usepackage{amsmath}
\usepackage{tikz}
\usepackage[margin=1in]{geometry}
\usepackage[backend=biber, style=numeric]{biblatex}
%\usepackage{hyperref}
\usepackage{bookmark}
\usepackage{enumitem}
\usepackage{amsmath}
\usepackage{listings}
\lstset{language=Python, basicstyle=\ttfamily\small, breaklines=true}
\lstset{escapeinside={(*@}{@*)}}
\usepackage{fontspec}
\setmainfont{Arial}
\newfontfamily\stylishfont{Noteworthy}
%\newfontfamily\stylishfont{Zapfino}
\addbibresource{references.bib}
\usetikzlibrary{calc}


\lstdefinestyle{pidstyle}{
    basicstyle=\ttfamily\footnotesize,
    breaklines=true,
    escapechar=\#, % Define escape character for inline LaTeX commands
    linewidth=\textwidth,
    basicstyle=\ttfamily\scriptsize
}

\renewcommand{\maketitle}{%
  \begin{leftmark}
    \vspace*{\baselineskip} % Add a bit of vertical space

%    \includegraphics[width=4cm]{example-image-a} % Add an image before the title. you will need to replace the image path with your own

%    \vspace{0.5cm} % Add vertical space before title

    \textbf{\fontsize{18}{36}\selectfont \thetitle} % Font Size and Bold Title

     \vspace{0.05cm} % Add vertical space before subtitle
%    \textit{\Large \theauthor}  % Subtitle / Author
    \vspace{\baselineskip} % Add vertical space after subtitle
     \rule{\textwidth}{0.4pt} % Add a horizontal line

   \end{leftmark}
%    \thispagestyle{empty} % Prevent header/footer on the title page
}


% Section Formatting
\titleformat{\section}
  {\normalfont\fontsize{18}{22}\bfseries} % Font and style
  {\thesection}         % Section number
  {1em}                   % Horizontal space after section number
  {}                     % Code before the section name
  []                     % Code after the section name

\titleformat{\subsection}
  {\normalfont\fontsize{14}{18}\bfseries} % Font and style
  {\thesubsection}         % Subsection number
  {1em}                   % Horizontal space after subsection number
  {}                     % Code before the subsection name
  []                     % Code after the subsection name

\setlength{\parindent}{0pt}

\title{Computing platforms (Spring 2025)\newline
week 5}
\author{Juha-Pekka Heikkilä}



\pagestyle{fancy}
\fancyhf{}

\renewcommand{\headrulewidth}{0pt}

\newcommand{\footerline}{\makebox[\textwidth]{\hrulefill}}

\newcommand{\footercontent}{%
    \begin{tabular}{@{}l@{}}
        \footerline \\
        \leftmark \hfill \rlap{\thepage}
    \end{tabular}
}

\fancyfoot[C]{\footercontent}


\newcommand{\exercise}[1]{
    \section*{Exercise #1}
    \markboth{Exercise #1}{}
}


\begin{document}
\maketitle


\exercise{1}
\textbf{Synchronous and asynchronous communication.} In
synchronous communication both communicating parties are
expected to be present at the same time, while
in asynchronous communication they can run separate times.


\begin{enumerate}[label=\textbf{\makebox[1cm][l]{\Huge\text{(\stylishfont\alph*)}}}, leftmargin=!, labelindent=0pt]
  \item What if your application logic needs synchronous
  communication, but the environment only provide asynchronous
  communication? How can you use asynchronous communication to
  implement synchronous communication?\newline

  \begin{itemize}
    \item \textbf{Sending a Request:} The sender dispatches
    a message asynchronously to the receiver.
    \item \textbf{Attaching an Identifier:} Include a unique
    identifier with the message so that the corresponding
    response can be recognized.
    \item \textbf{Blocking for the Response:} The sender then
    waits (i.e., blocks) until it receives a reply matching
    the identifier. This wait can be implemented using
    mechanisms such as callbacks or condition variables.
    \item \textbf{Resuming Execution:} Once the response is
    received, the sender unblocks and continues execution.
  \end{itemize}


This approach creates request-response (synchronous) pattern
over asynchronous channel, ensuring the sender does not proceed
until the required information is available.\newline




  \item Think about a reverse situation.
  You need asynchronous communication, but only synchronous
  communication is available? How can you implement
  asynchronous communication using synchronous communication?\newline

  \begin{itemize}
    \item \textbf{Using Concurrency:} Spawn separate thread or process
    to handle the synchronous communication. Main thread can continue
    execution without waiting.
    \item \textbf{Buffering Messages:} Create local message queue or
    buffer where outgoing messages are stored. Dedicated worker thread
    performs synchronous send/receive operations behalf of main
    application.
    \item \textbf{Event-Driven Notification:} On completing synchronous
    operation, worker thread notify main thread (e.g., via 
    event or callback mechanism) that communication is complete,
    thus decoupling the communication from main control flow.
  \end{itemize}

By offloading the blocking operations to separate threads and using
buffering, overall system behaves in an asynchronous manner despite
underlying synchronous primitives.

\end{enumerate}



\newpage

\exercise{2}
You have an existing program that uses pipes (named or unnamed) as communication mechanism
between two processes. Now you need to move one of the processes to another computer and
you need to change the communication mechanism, because pipes can be used only within one
computer. Which of the mechanisms (shared memory, signal, file, socket, message queue, remote
procedure call) would you be able to use and why? Please also give reasons for the rest, why
they would not be suitable in this case.

\subsection*{Suitable mechanisms}
\begin{itemize}
    \item \textbf{Socket:} Sockets are designed for network communication.
    They allow data to be transmitted between processes on
    different computers reliably and efficiently.
    \item \textbf{Message Queue:} Distributed message queue systems
    enable asynchronous communication across machines. They provide
    buffering, reliability, and decoupling of processes.
    \item \textbf{Remote Procedure Call (RPC):} RPC abstracts network
    communication details and allows a process to invoke functions
    on a remote machine. It encapsulates the network protocols and
    provides a clean interface for inter-machine communication.
\end{itemize}

\subsection*{Not suitable mechanisms:}

\begin{itemize}
    \item \textbf{Shared Memory:} Shared memory is based on idea
    of multiple processes accessing common memory space, which is
    only feasible on the same physical computer.
    \item \textbf{Signal:} Signals are primarily used for simple
    notifications or process control within a single machine.
    They do not carry amount of data or offer reliability needed
    for communication across different machines.
    \item \textbf{File:} While files can be used for communication
    via a shared network file system, this approach is prone
    to synchronization issues. File-based communication is less
    suitable for real-time or interactive inter-process communication.
\end{itemize}




\newpage

\exercise{3}
\textbf{Race condition.} Alice and Bob come to run 4000 meters on a running track that is 400 meters
long. The python code “race-condition.py” shows how they are counting the laps. Download the
code separately from Moodle.

\begin{enumerate}[label=\textbf{\makebox[1cm][l]{\Huge\text{(\stylishfont\alph*)}}}, leftmargin=!, labelindent=0pt]
  \item Run the code several times on your computer.
What do you observe?\newline

The Python program creates two threads (Alice and Bob) both running
10 iterations. In each iteration the thread:
\begin{itemize}
    \item Read shared variable \textbf{laps}.
    \item Compute \textbf{mylap = laps + 1}.
    \item Assign \textbf{laps = mylap}.
\end{itemize}

Because there is no synchronization, two threads can interleave
their operations in a way that leads to \textbf{lost updates}.

  
  \item Determine the proper lower and upper bound on the final value
  of the shared variable laps when this concurrent program terminates.
  Assume threads can execute at any relative speed.\newline

When you run the code several times, you may observe that:
\begin{itemize}
    \item The printed lap numbers may not increase sequentially as expected.
    \item The final value of \textbf{laps} is not always 20 (i.e. 10
    iterations per thread). It often appears to be lower.
\end{itemize}
This is a manifestation of a \emph{race condition} where both
threads read and update the shared variable without coordination,
causing some increments to be “lost.”


\paragraph{Upper Bound:} 
The best-case scenario occurs when the threads never interfere
(i.e. their operations happen sequentially without overlap).
In that case every increment is preserved and
\[
\text{Final value} = 10 + 10 = 20.
\]

\paragraph{Lower Bound:}
A worst-case scenario is possible if a thread, due to scheduling delays, reads
\emph{stale} value before any updates occur and then later writes that
stale increment, thereby \emph{overwriting} progress made by other thread.
\begin{itemize}
    \item Suppose Alice starts first iteration and reads
    \textbf{laps} when it is \textbf{0} but gets delayed before writing.
    \item Meanwhile Bob performs all of his $10$ iterations correctly,
    updating \textbf{laps} from $0$ to $10$.
    \item Then Alice resumes and, still holding stale value from
    her first iteration, writes \textbf{laps = 0 + 1 = 1}.
\end{itemize}
Even if only one such \emph{rollback} occurs, it can drastically
reduce the final count. In the extreme, similar delays in several
iterations could cause many increments to be lost. By careful
interleaving it is possible for final value to be as low as
\textbf{1} (i.e. only one effective update is preserved).

\paragraph{Conclusion:}
\[
1 \le \texttt{laps} \le 20.
\]
One might expect "pairing" of increments to yield lower bound
of 10, but possibility of \emph{stale reads} and \emph{rollbacks}
means that in worst-case final value can be as low as 1.


\emph{Note: Printed output is not necessarily precise reflection
of actual execution order, since print statements are unsynchronized
and may interleave arbitrarily with thread operations.}

\end{enumerate}



\newpage

\exercise{4}
\textbf{Locks and condition variables.} In this exercise we will
continue with the python code of Alice and Bob counting the laps
they are running. Here we will modify it so that Alice and Bob
synchronize with each other properly.

\begin{enumerate}[label=\textbf{\makebox[1cm][l]{\Huge\text{(\stylishfont\alph*)}}}, leftmargin=!, labelindent=0pt]

  \item Modify the python code so that the critical sections are
  protected properly by using locks. Make your solution as simple
  as possible but still correct in all possible scenarios. \newline



\begin{lstlisting}
  from threading import Thread, current_thread, (*@\colorbox{yellow}{Lock}@*)

  # Alice and Bob will both run n laps
  n=10
  # Total number of laps
  laps=0
  (*@\colorbox{yellow}{lock = Lock()}@*)
  
  # Function executed by both Alice and Bob threads
  def runlaps():
      global laps
      name = current_thread().name    # Is this Alice or Bob?
      for _ in range(n):
          (*@\colorbox{yellow}{with lock:}@*)
              # Which lap is starting?
              mylap = laps+1
              print(name + " starting lap: " + str(mylap))
              # Store the lap number that was finished
              laps = mylap
              print(name + " finished lap: " + str(laps))
  
  # Create the threads for Alice and Bob
  alice_thread = Thread(target=runlaps, name="Alice")
  bob_thread = Thread(target=runlaps, name="Bob")
  
  # Start the threads
  alice_thread.start()
  bob_thread.start()
  
  # Wait for Alice and Bob threads to finish
  alice_thread.join()
  bob_thread.join()
  
  # Print the total number of laps
  print("Total laps: " + str(laps))  
\end{lstlisting}

\newpage
\item After each lap, Alice waits for Bob to catch up with her
(on the same lap, so that both have run equal number of laps).
If Bob happens to be running ahead, he will keep on running
and lets Alice try to catch him. So Bob just runs for his 10 laps.
On a good day Bob could pass Alice many times and finish his run
when Alice still has many laps to go. Use condition variables
to have Alice wait for Bob after each lap.\newline


To implement this we need separate counters for Alice and Bob.
We also use a \texttt{Condition} variable, which is associated
with a lock, to allow Alice to wait until Bob's counter catches up.

\begin{lstlisting}
  from threading import Thread, current_thread, (*@\colorbox{yellow}{Condition}@*)

  # Alice and Bob will both run n laps
  n=10
  # Separate lap counters for Alice and Bob
  (*@\colorbox{yellow}{alice\_laps = 0}@*)
  (*@\colorbox{yellow}{bob\_laps = 0}@*)
  
  # Condition variable with its internal lock
  cond = Condition()
  
  # Function executed by Alice
  def runlaps_alice():
      global alice_laps, bob_laps
      for _ in range(n):
          (*@\colorbox{yellow}{with cond:}@*)
              # Which lap is starting?
              alice_laps += 1
              print("Alice finished lap: " + str(alice_laps))
              (*@\colorbox{yellow}{cond.notify\_all()}@*)
              (*@\colorbox{yellow}{while bob\_laps < alice\_laps:}@*)
                  (*@\colorbox{yellow}{cond.wait()}@*)
  
  # Function executed by Bob
  def runlaps_bob():
      global bob_laps
      for _ in range(n):
          (*@\colorbox{yellow}{with cond:}@*)
              bob_laps += 1
              print("Bob finished lap: " + str(bob_laps))
              (*@\colorbox{yellow}{cond.notify\_all()}@*)
  
  # Create the threads for Alice and Bob
  alice_thread = Thread(target=runlaps_alice, name="Alice")
  bob_thread = Thread(target=runlaps_bob, name="Bob")
  
  # Start the threads
  alice_thread.start()
  bob_thread.start()
  
  # Wait for Alice and Bob threads to finish
  alice_thread.join()
  bob_thread.join()
  
  # Print the total number of laps
  print("Final laps: Alice = " + str(alice_laps) + ", Bob = " + str(bob_laps))
  \end{lstlisting}


\end{enumerate}








\newpage

\exercise{5}






\newpage
\printbibliography
\end{document}
