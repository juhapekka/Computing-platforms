% !TEX program = xelatex
\documentclass{article}
\usepackage[utf8]{inputenc}
\usepackage{graphicx}
\usepackage{titling}
\usepackage{titlesec}
\usepackage{booktabs}
\usepackage{fancyhdr}
\usepackage{lipsum}
\usepackage{comment}
\usepackage{enumitem}
\usepackage{listings}
\usepackage{xcolor}
\usepackage{longtable}
%\usepackage{cite}
\usepackage{pgfgantt}
\usepackage{amsmath}
\usepackage{tikz}
\usepackage[margin=1in]{geometry}
\usepackage[backend=biber, style=numeric]{biblatex}
%\usepackage{hyperref}
\usepackage{bookmark}
\usepackage{enumitem}
\usepackage{amsmath}
\usepackage{fontspec}
\setmainfont{Arial}
\newfontfamily\stylishfont{Noteworthy}
%\newfontfamily\stylishfont{Zapfino}
\addbibresource{references.bib}
\usetikzlibrary{calc}


\lstdefinestyle{pidstyle}{
    basicstyle=\ttfamily\footnotesize,
    breaklines=true,
    escapechar=\#, % Define escape character for inline LaTeX commands
    linewidth=\textwidth,
    basicstyle=\ttfamily\scriptsize
}

\renewcommand{\maketitle}{%
  \begin{leftmark}
    \vspace*{\baselineskip} % Add a bit of vertical space

%    \includegraphics[width=4cm]{example-image-a} % Add an image before the title. you will need to replace the image path with your own

%    \vspace{0.5cm} % Add vertical space before title

    \textbf{\fontsize{18}{36}\selectfont \thetitle} % Font Size and Bold Title

     \vspace{0.05cm} % Add vertical space before subtitle
%    \textit{\Large \theauthor}  % Subtitle / Author
    \vspace{\baselineskip} % Add vertical space after subtitle
     \rule{\textwidth}{0.4pt} % Add a horizontal line

   \end{leftmark}
%    \thispagestyle{empty} % Prevent header/footer on the title page
}


% Section Formatting
\titleformat{\section}
  {\normalfont\fontsize{18}{22}\bfseries} % Font and style
  {\thesection}         % Section number
  {1em}                   % Horizontal space after section number
  {}                     % Code before the section name
  []                     % Code after the section name

\titleformat{\subsection}
  {\normalfont\fontsize{14}{18}\bfseries} % Font and style
  {\thesubsection}         % Subsection number
  {1em}                   % Horizontal space after subsection number
  {}                     % Code before the subsection name
  []                     % Code after the subsection name

\setlength{\parindent}{0pt}

\title{Computing platforms (Spring 2025)\newline
week 5}
\author{Juha-Pekka Heikkilä}



\pagestyle{fancy}
\fancyhf{}

\renewcommand{\headrulewidth}{0pt}

\newcommand{\footerline}{\makebox[\textwidth]{\hrulefill}}

\newcommand{\footercontent}{%
    \begin{tabular}{@{}l@{}}
        \footerline \\
        \leftmark \hfill \rlap{\thepage}
    \end{tabular}
}

\fancyfoot[C]{\footercontent}


\newcommand{\exercise}[1]{
    \section*{Exercise #1}
    \markboth{Exercise #1}{}
}


\begin{document}
\maketitle


\exercise{1}
\textbf{Synchronous and asynchronous communication.} In
synchronous communication both communicating parties are
expected to be present at the same time, while
in asynchronous communication they can run separate times.


\begin{enumerate}[label=\textbf{\makebox[1cm][l]{\Huge\text{(\stylishfont\alph*)}}}, leftmargin=!, labelindent=0pt]
  \item What if your application logic needs synchronous
  communication, but the environment only provide asynchronous
  communication? How can you use asynchronous communication to
  implement synchronous communication?\newline

  \begin{itemize}
    \item \textbf{Sending a Request:} The sender dispatches
    a message asynchronously to the receiver.
    \item \textbf{Attaching an Identifier:} Include a unique
    identifier with the message so that the corresponding
    response can be recognized.
    \item \textbf{Blocking for the Response:} The sender then
    waits (i.e., blocks) until it receives a reply matching
    the identifier. This wait can be implemented using
    mechanisms such as callbacks or condition variables.
    \item \textbf{Resuming Execution:} Once the response is
    received, the sender unblocks and continues execution.
  \end{itemize}


This approach creates request-response (synchronous) pattern
over asynchronous channel, ensuring the sender does not proceed
until the required information is available.\newline




  \item Think about a reverse situation.
  You need asynchronous communication, but only synchronous
  communication is available? How can you implement
  asynchronous communication using synchronous communication?\newline

  \begin{itemize}
    \item \textbf{Using Concurrency:} Spawn separate thread or process
    to handle the synchronous communication. Main thread can continue
    execution without waiting.
    \item \textbf{Buffering Messages:} Create local message queue or
    buffer where outgoing messages are stored. Dedicated worker thread
    performs synchronous send/receive operations behalf of main
    application.
    \item \textbf{Event-Driven Notification:} On completing synchronous
    operation, worker thread notify main thread (e.g., via 
    event or callback mechanism) that communication is complete,
    thus decoupling the communication from main control flow.
  \end{itemize}

By offloading the blocking operations to separate threads and using
buffering, overall system behaves in an asynchronous manner despite
underlying synchronous primitives.

\end{enumerate}



\newpage

\exercise{2}


\newpage
\printbibliography
\end{document}
