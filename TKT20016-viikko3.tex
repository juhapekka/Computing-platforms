\documentclass{article}
\usepackage[utf8]{inputenc}
\usepackage{graphicx}
\usepackage{titling}
\usepackage{titlesec}
\usepackage{booktabs}
\usepackage{fancyhdr}
\usepackage{lipsum}
\usepackage{comment}
\usepackage{enumitem}
\usepackage{listings}
\usepackage{xcolor}
\usepackage{longtable}
\usepackage{cite}
\usepackage{pgfgantt}
\usepackage{amsmath}
\usepackage{tikz}
\usepackage[margin=1in]{geometry}
\usetikzlibrary{calc}


\lstdefinestyle{pidstyle}{
    basicstyle=\ttfamily\footnotesize,
    breaklines=true,
    escapechar=\#, % Define escape character for inline LaTeX commands
    linewidth=\textwidth,
    basicstyle=\ttfamily\scriptsize
}

\renewcommand{\maketitle}{%
  \begin{leftmark}
    \vspace*{\baselineskip} % Add a bit of vertical space

%    \includegraphics[width=4cm]{example-image-a} % Add an image before the title. you will need to replace the image path with your own

%    \vspace{0.5cm} % Add vertical space before title

    \textbf{\fontsize{18}{36}\selectfont \thetitle} % Font Size and Bold Title

     \vspace{0.05cm} % Add vertical space before subtitle
%    \textit{\Large \theauthor}  % Subtitle / Author
    \vspace{\baselineskip} % Add vertical space after subtitle
     \rule{\textwidth}{0.4pt} % Add a horizontal line

   \end{leftmark}
%    \thispagestyle{empty} % Prevent header/footer on the title page
}


% Section Formatting
\titleformat{\section}
  {\normalfont\fontsize{18}{22}\bfseries} % Font and style
  {\thesection}         % Section number
  {1em}                   % Horizontal space after section number
  {}                     % Code before the section name
  []                     % Code after the section name

\titleformat{\subsection}
  {\normalfont\fontsize{14}{18}\bfseries} % Font and style
  {\thesubsection}         % Subsection number
  {1em}                   % Horizontal space after subsection number
  {}                     % Code before the subsection name
  []                     % Code after the subsection name

\setlength{\parindent}{0pt}

\title{Computing platforms (Spring 2025)\newline
week 3}
\author{Juha-Pekka Heikkilä}



\pagestyle{fancy}
\fancyhf{}

\renewcommand{\headrulewidth}{0pt}

\newcommand{\footerline}{\makebox[\textwidth]{\hrulefill}}

\newcommand{\footercontent}{%
    \begin{tabular}{@{}l@{}}
        \footerline \\
        \leftmark \hfill \rlap{\thepage}
    \end{tabular}
}

\fancyfoot[C]{\footercontent}


\newcommand{\exercise}[1]{
    \section*{Exercise #1}
    \markboth{Exercise #1}{}
}



\begin{document}
\maketitle


\exercise{1}
\section*{Address Translation: Segmentation}

32-bit address space, leftmost 4 bits represent segment
number and remaining 28 bits offset within that segment.
The segmentation table for a process is as follows:

\bigskip
\begin{center}
\begin{tabular}{|c|c|c|}
\hline
\textbf{Index} & \textbf{Base Address} & \textbf{Segment Length} \\
\hline
0 & \texttt{0x00001000} & \texttt{0x0500} \\
1 & \texttt{0x00020000} & \texttt{0x5000} \\
2 & \texttt{0x00030000} & \texttt{0x10000} \\
3 & \texttt{0x00045000} & \texttt{0x5000} \\
\hline
\end{tabular}
\end{center}
\bigskip

\subsection*{(a) Translating Virtual Address \texttt{0x20001111}}

The virtual address can be split into:

\begin{center}
\begin{tabular}{|c|c|}
\hline
\textbf{Segment Number} & \textbf{Offset} \\
\hline
\texttt{0x2} (Segment 2) & \texttt{0x0001111} \\
\hline
\end{tabular}
\end{center}

For segment 2:

\begin{center}
\begin{tabular}{|c|c|c|}
\hline
\textbf{Segment} & \textbf{Base Address} & \textbf{Segment Length} \\
\hline
2 & \texttt{0x00030000} & \texttt{0x10000} \\
\hline
\end{tabular}
\end{center}

Since \texttt{0x0001111} is within the segment length \texttt{0x10000},
the physical address is:

\[
\text{Physical Address} = \texttt{Base Address} + \texttt{Offset} = \texttt{0x00030000} + \texttt{0x0001111} = \texttt{0x00031111}
\]

\bigskip
\textbf{Answer for (a):} The virtual address \(\texttt{0x20001111}\)
translates to the physical address \(\texttt{0x00031111}\).





\newpage
\subsection*{(b) Translating Virtual Address \texttt{0x00001000}}

The virtual address can be split into:

\begin{center}
\begin{tabular}{|c|c|}
\hline
\textbf{Segment Number} & \textbf{Offset} \\
\hline
\texttt{0x0} (Segment 0) & \texttt{0x0001000} \\
\hline
\end{tabular}
\end{center}

For segment 0:

\begin{center}
\begin{tabular}{|c|c|c|}
\hline
\textbf{Segment} & \textbf{Base Address} & \textbf{Segment Length} \\
\hline
0 & \texttt{0x00001000} & \texttt{0x0500} \\
\hline
\end{tabular}
\end{center}

Since the offset \texttt{0x1000} exceeds the segment length \texttt{0x0500},
 segmentation fault occurs.

\bigskip
\textbf{Answer for (b):} The virtual address \(\texttt{0x00001000}\)
is invalid because the offset exceeds the segment’s size,
causing segmentation fault.







\newpage
\exercise{2}

\section*{Address Translation: Paging}

We have a system with:
\begin{itemize}
  \item 32-bit virtual addresses
  \item 4\,KB pages (\(4\,\text{KB} = 2^{12}\,\text{bytes}\))
  \item A single-level page table
\end{itemize}

\subsection*{(a) Layout of Virtual Addresses}

Since each page is \(2^{12}\) bytes in size, the \textbf{offset}
occupies the lower 12 bits of the address. The remaining upper
20 bits specify the \textbf{page number}. Thus, a 32-bit virtual
address is divided as follows:

\[
\underbrace{\text{Page Number}}_{20\text{ bits}}
\;\big\vert\;
\underbrace{\text{Offset}}_{12\text{ bits}}
\]

\[
\texttt{[ 5 hex digits ] [ 3 hex digits ]}
\]

\subsection*{(b) Translating Virtual Address \texttt{0x00001234}}

Assume a process has single-level page table shown below:

\bigskip
\begin{center}
\begin{tabular}{|c|c|}
\hline
\textbf{Index} & \textbf{Frame Number} \\
\hline
0 & 3 \\
1 & 2 \\
2 & 7 \\
3 & 1 \\
\hline
\end{tabular}
\end{center}
\bigskip

\paragraph{Step 1: Split the Virtual Address.}

\[
\texttt{0x00001234} \quad \longrightarrow \quad
\begin{cases}
\text{Page Number} = 1 \\
\text{Offset} = 0x234
\end{cases}
\]

\paragraph{Step 2: Look Up the Frame.}

Using \(\text{page number} = 1 \longrightarrow \text{Frame Number} = 2\)


\paragraph{Step 3: Construct the Physical Address.}

Each frame is also \(4\,\text{KB} = 2^{12}\,\text{bytes}\) hence frame number~2 starts at
\[
\underbrace{2 \times 2^{12}}_{\text{Frame Number} \times \text{Frame Size}}
= 2 \times 0x1000 
= \texttt{0x2000}.
\]

and add offset 0x234:
\[
\text{Physical Address} = \texttt{0x00002000} + \texttt{0x00000234} = \texttt{0x00002234}.
\]



\newpage
\bibliographystyle{plain}
\bibliography{references}
\end{document}