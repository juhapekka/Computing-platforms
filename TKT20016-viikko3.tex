\documentclass{article}
\usepackage[utf8]{inputenc}
\usepackage{graphicx}
\usepackage{titling}
\usepackage{titlesec}
\usepackage{booktabs}
\usepackage{fancyhdr}
\usepackage{lipsum}
\usepackage{comment}
\usepackage{enumitem}
\usepackage{listings}
\usepackage{xcolor}
\usepackage{longtable}
\usepackage{cite}
\usepackage{pgfgantt}
\usepackage{amsmath}
\usepackage{tikz}
\usepackage[margin=1in]{geometry}
\usetikzlibrary{calc}


\lstdefinestyle{pidstyle}{
    basicstyle=\ttfamily\footnotesize,
    breaklines=true,
    escapechar=\#, % Define escape character for inline LaTeX commands
    linewidth=\textwidth,
    basicstyle=\ttfamily\scriptsize
}

\renewcommand{\maketitle}{%
  \begin{leftmark}
    \vspace*{\baselineskip} % Add a bit of vertical space

%    \includegraphics[width=4cm]{example-image-a} % Add an image before the title. you will need to replace the image path with your own

%    \vspace{0.5cm} % Add vertical space before title

    \textbf{\fontsize{18}{36}\selectfont \thetitle} % Font Size and Bold Title

     \vspace{0.05cm} % Add vertical space before subtitle
%    \textit{\Large \theauthor}  % Subtitle / Author
    \vspace{\baselineskip} % Add vertical space after subtitle
     \rule{\textwidth}{0.4pt} % Add a horizontal line

   \end{leftmark}
%    \thispagestyle{empty} % Prevent header/footer on the title page
}


% Section Formatting
\titleformat{\section}
  {\normalfont\fontsize{18}{22}\bfseries} % Font and style
  {\thesection}         % Section number
  {1em}                   % Horizontal space after section number
  {}                     % Code before the section name
  []                     % Code after the section name

\titleformat{\subsection}
  {\normalfont\fontsize{14}{18}\bfseries} % Font and style
  {\thesubsection}         % Subsection number
  {1em}                   % Horizontal space after subsection number
  {}                     % Code before the subsection name
  []                     % Code after the subsection name

\setlength{\parindent}{0pt}

\title{Computing platforms (Spring 2025)\newline
week 3}
\author{Juha-Pekka Heikkilä}



\pagestyle{fancy}
\fancyhf{}

\renewcommand{\headrulewidth}{0pt}

\newcommand{\footerline}{\makebox[\textwidth]{\hrulefill}}

\newcommand{\footercontent}{%
    \begin{tabular}{@{}l@{}}
        \footerline \\
        \leftmark \hfill \rlap{\thepage}
    \end{tabular}
}

\fancyfoot[C]{\footercontent}


\newcommand{\exercise}[1]{
    \section*{Exercise #1}
    \markboth{Exercise #1}{}
}



\begin{document}
\maketitle


\exercise{1}
\section*{Address Translation: Segmentation}

32-bit address space, leftmost 4 bits represent segment
number and remaining 28 bits offset within that segment.
The segmentation table for a process is as follows:

\bigskip
\begin{center}
\begin{tabular}{|c|c|c|}
\hline
\textbf{Index} & \textbf{Base Address} & \textbf{Segment Length} \\
\hline
0 & \texttt{0x00001000} & \texttt{0x0500} \\
1 & \texttt{0x00020000} & \texttt{0x5000} \\
2 & \texttt{0x00030000} & \texttt{0x10000} \\
3 & \texttt{0x00045000} & \texttt{0x5000} \\
\hline
\end{tabular}
\end{center}
\bigskip

\subsection*{(a) Translating Virtual Address \texttt{0x20001111}}

The virtual address can be split into:

\begin{center}
\begin{tabular}{|c|c|}
\hline
\textbf{Segment Number} & \textbf{Offset} \\
\hline
\texttt{0x2} (Segment 2) & \texttt{0x0001111} \\
\hline
\end{tabular}
\end{center}

For segment 2:

\begin{center}
\begin{tabular}{|c|c|c|}
\hline
\textbf{Segment} & \textbf{Base Address} & \textbf{Segment Length} \\
\hline
2 & \texttt{0x00030000} & \texttt{0x10000} \\
\hline
\end{tabular}
\end{center}

Since \texttt{0x0001111} is within the segment length \texttt{0x10000},
the physical address is:

\[
\text{Physical Address} = \texttt{Base Address} + \texttt{Offset} = \texttt{0x00030000} + \texttt{0x0001111} = \texttt{0x00031111}
\]

\bigskip
\textbf{Answer for (a):} The virtual address \(\texttt{0x20001111}\)
translates to the physical address \(\texttt{0x00031111}\).





\newpage
\subsection*{(b) Translating Virtual Address \texttt{0x00001000}}

The virtual address can be split into:

\begin{center}
\begin{tabular}{|c|c|}
\hline
\textbf{Segment Number} & \textbf{Offset} \\
\hline
\texttt{0x0} (Segment 0) & \texttt{0x0001000} \\
\hline
\end{tabular}
\end{center}

For segment 0:

\begin{center}
\begin{tabular}{|c|c|c|}
\hline
\textbf{Segment} & \textbf{Base Address} & \textbf{Segment Length} \\
\hline
0 & \texttt{0x00001000} & \texttt{0x0500} \\
\hline
\end{tabular}
\end{center}

Since the offset \texttt{0x1000} exceeds the segment length \texttt{0x0500},
 segmentation fault occurs.

\bigskip
\textbf{Answer for (b):} The virtual address \(\texttt{0x00001000}\)
is invalid because the offset exceeds the segment’s size,
causing segmentation fault.







\newpage
\exercise{2}

\section*{Address Translation: Paging}

We have a system with:
\begin{itemize}
  \item 32-bit virtual addresses
  \item 4\,KB pages (\(4\,\text{KB} = 2^{12}\,\text{bytes}\))
  \item A single-level page table
\end{itemize}

\subsection*{(a) Layout of Virtual Addresses}

Since each page is \(2^{12}\) bytes in size, the \textbf{offset}
occupies the lower 12 bits of the address. The remaining upper
20 bits specify the \textbf{page number}. Thus, a 32-bit virtual
address is divided as follows:

\[
\underbrace{\text{Page Number}}_{20\text{ bits}}
\;\big\vert\;
\underbrace{\text{Offset}}_{12\text{ bits}}
\]

\[
\texttt{[ 5 hex digits ] [ 3 hex digits ]}
\]

\subsection*{(b) Translating Virtual Address \texttt{0x00001234}}

Assume a process has single-level page table shown below:

\bigskip
\begin{center}
\begin{tabular}{|c|c|}
\hline
\textbf{Index} & \textbf{Frame Number} \\
\hline
0 & 3 \\
1 & 2 \\
2 & 7 \\
3 & 1 \\
\hline
\end{tabular}
\end{center}
\bigskip

\paragraph{Step 1: Split the Virtual Address.}

\[
\texttt{0x00001234} \quad \longrightarrow \quad
\begin{cases}
\text{Page Number} = 1 \\
\text{Offset} = 0x234
\end{cases}
\]

\paragraph{Step 2: Look Up the Frame.}

Using \(\text{page number} = 1 \longrightarrow \text{Frame Number} = 2\)


\paragraph{Step 3: Construct the Physical Address.}

Each frame is also \(4\,\text{KB} = 2^{12}\,\text{bytes}\) hence frame number~2 starts at
\[
\underbrace{2 \times 2^{12}}_{\text{Frame Number} \times \text{Frame Size}}
= 2 \times 0x1000 
= \texttt{0x2000}.
\]

and add offset 0x234:
\[
\text{Physical Address} = \texttt{0x00002000} + \texttt{0x00000234} = \texttt{0x00002234}.
\]






\newpage
\exercise{3}


\section*{Virtual Memory with Paging and TLB}

We have:
\begin{itemize}
  \item 32-bit virtual addresses
  \item 4\,KB pages (\(4\,\text{KB} = 2^{12}\,\text{bytes}\))
  \item A single-level page table and a Translation Lookaside Buffer (TLB)
\end{itemize}

\subsection*{Page Table and TLB Contents}

single-level page table:\newline

\begin{tabular}{|c|c|c|c|}
  \hline
  \multicolumn{4}{|c|}{{\bf Page table}} \\
\hline
\textbf{Index} & \textbf{Frame Number} & \textbf{M} & \textbf{P} \\
\hline
0 & 3 & 0 & 1 \\
1 & 2 & 0 & 1 \\
2 & 7 & 0 & 0 \\
3 & 1 & 1 & 1 \\
\hline
\end{tabular}
\begin{tabular}{|c|c|c|c|c|}
  \hline
  \multicolumn{5}{|c|}{{\bf TLB}} \\
\hline
\textbf{Index} & \textbf{Page \#} & \textbf{Frame \#} & \textbf{M} & \textbf{P} \\
\hline
0 & 3 & 1 & 1 & 1 \\
1 & 0 & 3 & 0 & 1 \\
2 &   &   &   & 0 \\
3 &   &   &   & 0 \\
\hline
\end{tabular}
\begin{itemize}
  \item \texttt{M = 1} indicates the cached page is modified.
  \item \texttt{P = 1} indicates the TLB entry is valid (present).
  \item Entries 2 and 3 are either empty or invalid (\texttt{P = 0}).
\end{itemize}

\subsection*{Address Format (32-bit, 4\,KB Pages)}

A 32-bit address is split into:
\[
\underbrace{\text{Page Number}}_{20 \text{ bits}}
\;\big\vert\;
\underbrace{\text{Offset}}_{12 \text{ bits}}
\]
Hence, the lower 12 bits (3 hex digits) are the \emph{offset}, and the upper 20 bits are the \emph{page number}.

\newpage




\subsection*{(a) Translating Address \texttt{0x00001123}}

\paragraph{Step 1: Split the Address.}
\[
\texttt{0x00001123}
\]
\begin{itemize}
  \item Offset = \texttt{0x123} (the lowest 12 bits)
  \item Page Number = \texttt{0x1} (the upper 20 bits)
\end{itemize}

\paragraph{Step 2: TLB Lookup.}
We check whether the TLB has an entry for Page~1.
\begin{itemize}
  \item Index~0 has Page~3.
  \item Index~1 has Page~0.
  \item Index~2,3 are invalid.
\end{itemize}
No match $\Rightarrow$ \textbf{TLB miss}.

\paragraph{Step 3: Page Table Lookup.}
From the Page Table:
\[
\text{Page~1} \longrightarrow \text{Frame~2}, \; M=0, \; P=1
\]
Since \(\texttt{P=1}\), the page is in memory (no page fault).

\paragraph{Step 4: Construct Physical Address.}
Frame~2 means the base of this frame is:
\[
2 \times 2^{12} = \texttt{0x2000}.
\]
Adding the offset \texttt{0x123}:
\[
\text{Physical Address} = \texttt{0x2000} + \texttt{0x0123} = \texttt{0x2123}.
\]

\paragraph{Step 5: Update TLB.}
We add a new TLB entry for Page~1:
\[
\begin{aligned}
  \text{Page No.} &= 1, \\
  \text{Frame No.} &= 2, \\
  M &= 0 \; (\text{no writes so far}), \\
  P &= 1 \; (\text{valid entry}).
\end{aligned}
\]
This typically occupies an empty TLB slot (say Index~2). 

\textbf{The virtual address \(\texttt{0x00001123}\) translates to \(\texttt{0x00002123}\). The TLB is updated to include Page~1 $\to$ Frame~2.}

\newpage




\subsection*{(b) Translating Address \texttt{0x00002333}}

\paragraph{Step 1: Split the Address.}
\[
\texttt{0x00002333}
\]
\begin{itemize}
  \item Offset = \texttt{0x333} (lowest 12 bits)
  \item Page Number = \texttt{0x2} (upper 20 bits)
\end{itemize}

\paragraph{Step 2: TLB Lookup.}
We check for Page~2 in the TLB:
\begin{itemize}
  \item Index~0 has Page~3,
  \item Index~1 has Page~0,
  \item Index~2 (updated above) has Page~1,
  \item Index~3 is invalid/empty.
\end{itemize}
No match $\Rightarrow$ \textbf{we got TLB miss again.}

\paragraph{Step 3: Page Table Lookup.}
From the Page Table:
\[
\text{Page~2} \longrightarrow \text{Frame~7}, \; M=0, \; P=0
\]
\(\texttt{P=0}\) means the page is not present in main memory $\Rightarrow$ \textbf{page fault}.

\paragraph{Step 4: Handle Page Fault.}
The OS must load Page~2 from disk into a free frame. Assume the OS chooses some free frame \(\texttt{F}\). After loading:
\[
\begin{aligned}
  &\text{Page~2 now mapped to Frame } \texttt{F}, \\
  &M=0 \;(\text{still no writes}), \quad P=1 \;(\text{now present}).
\end{aligned}
\]
The page table is updated accordingly:
\[
\text{Page Table}[2] \gets (\texttt{Frame~F}, M=0, P=1).
\]

\paragraph{Step 5: Construct Physical Address.}
Once the page is loaded, the physical base address is:
\[
\texttt{F} \times 2^{12},
\]
and we add the offset \texttt{0x333} to get the final physical address.

\paragraph{Step 6: Update TLB.}
We add a valid entry for Page~2 in the TLB:
\[
\text{Page~2} \longrightarrow \text{Frame~F},\, M=0,\, P=1.
\]

\textbf{The virtual address \(\texttt{0x00002333}\) causes a page
fault. After the OS loads Page~2 into some free frame \(\texttt{F}\),
the physical address is \(\texttt{(F * 0x1000) + 0x333}\). The page
table and TLB entries for Page~2 are updated to reflect
\(\texttt{Frame = F}\) and \(\texttt{P=1}\).}










\newpage
\exercise{4}



\section*{Page Replacement Comparison}

We have:
\begin{itemize}
  \item Main memory of 3 frames
  \item Reference stream: \texttt{7, 0, 1, 2, 0, 3, 0, 4, 2, 3, 0, 3, 2}
  \item We count page faults only after the first 3 references have 
  initialized the frames.
\end{itemize}

\bigskip

\renewcommand{\arraystretch}{1.2}
\setlength{\tabcolsep}{0.7em}

\begin{table}[h!]
\centering
\small
\begin{tabular}{l|ccccccccccccc}
\hline
\textbf{Ref.} & \textbf{7} & \textbf{0} & \textbf{1} & \textbf{2} & \textbf{0} & \textbf{3} & \textbf{0} & \textbf{4} & \textbf{2} & \textbf{3} & \textbf{0} & \textbf{3} & \textbf{2} \\
\hline

%---------------------------
% FIFO Row
%---------------------------
\textbf{FIFO} &

% Ref #1 => 7
\begin{tabular}{c}7\\-\\-\end{tabular} &

% Ref #2 => 0
\begin{tabular}{c}7\\0\\-\end{tabular} &

% Ref #3 => 1
\begin{tabular}{c}7\\0\\1\end{tabular} &

% Ref #4 => 2 (Fault #1 after init)
\begin{tabular}{c}2\\0\\1\\\textbf{F}\end{tabular} &

% Ref #5 => 0 (in frames)
\begin{tabular}{c}2\\0\\1\end{tabular} &

% Ref #6 => 3 (Fault #2)
\begin{tabular}{c}2\\3\\1\\\textbf{F}\end{tabular} &

% Ref #7 => 0 (Fault #3)
\begin{tabular}{c}2\\3\\0\\\textbf{F}\end{tabular} &

% Ref #8 => 4 (Fault #4)
\begin{tabular}{c}4\\3\\0\\\textbf{F}\end{tabular} &

% Ref #9 => 2 (Fault #5)
\begin{tabular}{c}4\\2\\0\\\textbf{F}\end{tabular} &

% Ref #10 => 3 (Fault #6)
\begin{tabular}{c}4\\2\\3\\\textbf{F}\end{tabular} &

% Ref #11 => 0 (Fault #7)
\begin{tabular}{c}0\\2\\3\\\textbf{F}\end{tabular} &

% Ref #12 => 3 (in frames)
\begin{tabular}{c}0\\2\\3\end{tabular} &

% Ref #13 => 2 (in frames)
\begin{tabular}{c}0\\2\\3\end{tabular}
\\

%---------------------------
% LRU Row
%---------------------------
\textbf{LRU} &

% #1 => 7
\begin{tabular}{c}7\\-\\-\end{tabular} &

% #2 => 0
\begin{tabular}{c}7\\0\\-\end{tabular} &

% #3 => 1
\begin{tabular}{c}7\\0\\1\end{tabular} &

% #4 => 2 (Fault #1)
\begin{tabular}{c}2\\0\\1\\\textbf{F}\end{tabular} &

% #5 => 0 (in frames)
\begin{tabular}{c}2\\0\\1\end{tabular} &

% #6 => 3 (Fault #2)
\begin{tabular}{c}2\\0\\3\\\textbf{F}\end{tabular} &

% #7 => 0 (in frames)
\begin{tabular}{c}2\\0\\3\end{tabular} &

% #8 => 4 (Fault #3)
\begin{tabular}{c}4\\0\\3\\\textbf{F}\end{tabular} &

% #9 => 2 (Fault #4)
\begin{tabular}{c}4\\0\\2\\\textbf{F}\end{tabular} &

% #10 => 3 (Fault #5)
\begin{tabular}{c}4\\3\\2\\\textbf{F}\end{tabular} &

% #11 => 0 (Fault #6)
\begin{tabular}{c}0\\3\\2\\\textbf{F}\end{tabular} &

% #12 => 3 (in frames)
\begin{tabular}{c}0\\3\\2\end{tabular} &

% #13 => 2 (in frames)
\begin{tabular}{c}0\\3\\2\end{tabular}
\\


%---------------------------
% Clock Row (schematic)
%---------------------------
\textbf{Clock} &

% #1 => 7
\begin{tabular}{c}7\\-\\-\end{tabular} &

% #2 => 0
\begin{tabular}{c}7\\0\\-\end{tabular} &

% #3 => 1
\begin{tabular}{c}7\\0\\1\end{tabular} &

% #4 => 2 (F?)
\begin{tabular}{c}2\\0\\1\\\textbf{F}\end{tabular} &

% #5 => 0
\begin{tabular}{c}2\\0\\1\end{tabular} &

% #6 => 3 (F?)
\begin{tabular}{c}2\\0\\3\\\textbf{F}\end{tabular} &

% #7 => 0
\begin{tabular}{c}2\\0\\3\end{tabular} &

% #8 => 4 (F?)
\begin{tabular}{c}4\\0\\3\\\textbf{F}\end{tabular} &

% #9 => 2 (F?)
\begin{tabular}{c}4\\2\\3\\\textbf{F}\end{tabular} &

% #10 => 3 (in frames)
\begin{tabular}{c}4\\2\\3\end{tabular} &

% #11 => 0 (F?)
\begin{tabular}{c}4\\2\\0\\\textbf{F}\end{tabular} &

% #12 => 3 (F?)
\begin{tabular}{c}4\\2\\3\end{tabular} &

% #13 => 2 (in frames)
\begin{tabular}{c}4\\2\\3\end{tabular}
\\

%---------------------------
% OPT Row
%---------------------------
\textbf{OPT} &

% #1 => 7
\begin{tabular}{c}7\\-\\-\end{tabular} &

% #2 => 0
\begin{tabular}{c}7\\0\\-\end{tabular} &

% #3 => 1
\begin{tabular}{c}7\\0\\1\end{tabular} &

% #4 => 2 (Fault #1)
\begin{tabular}{c}7\\0\\2\\\textbf{F}\end{tabular} &

% #5 => 0 (in frames)
\begin{tabular}{c}7\\0\\2\end{tabular} &

% #6 => 3 (Fault #2)
\begin{tabular}{c}7\\3\\2\\\textbf{F}\end{tabular} &

% #7 => 0 (in frames?)
\begin{tabular}{c}7\\3\\0\\\end{tabular} &

% #8 => 4 (Fault #3)
\begin{tabular}{c}2\\3\\0\\\textbf{F}\end{tabular} &

% #9 => 2 (in frames)
\begin{tabular}{c}2\\3\\0\end{tabular} &

% #10 => 3 (in frames)
\begin{tabular}{c}2\\3\\0\end{tabular} &

% #11 => 0 (Fault #4)
\begin{tabular}{c}2\\3\\4\\\textbf{F}\end{tabular} &

% #12 => 3 (in frames?)
\begin{tabular}{c}2\\3\\4\end{tabular} &

% #13 => 2 (in frames)
\begin{tabular}{c}2\\3\\4\end{tabular}
\\






\hline
\end{tabular}
\caption{Comparison of FIFO, LRU, OPT, and Clock. ``F'' marks a fault.}
\end{table}

\bigskip

\section*{Fault Counts and Miss Rates}

\begin{itemize}
  \item \textbf{FIFO:} 7 faults over 10 references $\Rightarrow$ 70\% miss rate.
  \item \textbf{LRU:} 6 faults $\Rightarrow$ 60\% miss rate.
  \item \textbf{Clock:} 4 or 5, depending on the initial hand and use-bit detail.
  \item \textbf{Optimal:} 5 faults if you consider extended reference stream 
        (\( \dots,1,2,0,1,7,0,1 \)) to decide which page is used farthest in the future.
\end{itemize}


\newpage
\bibliographystyle{plain}
\bibliography{references}
\end{document}