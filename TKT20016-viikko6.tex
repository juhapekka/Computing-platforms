% !TEX program = xelatex
\documentclass{article}
\usepackage[utf8]{inputenc}
\usepackage{graphicx}
\usepackage{titling}
\usepackage{titlesec}
\usepackage{booktabs}
\usepackage{fancyhdr}
\usepackage{lipsum}
\usepackage{comment}
\usepackage{enumitem}
\usepackage{xcolor}
\usepackage{longtable}
%\usepackage{cite}
\usepackage{pgfgantt}
\usepackage{amsmath, amssymb}
\usepackage{tikz}
\usepackage[margin=1in]{geometry}
\usepackage[backend=biber, style=numeric]{biblatex}
%\usepackage{hyperref}
\usepackage{bookmark}
\usepackage{enumitem}
\usepackage{amsmath}
\usepackage{listings}
\lstset{language=Python, basicstyle=\ttfamily\small, breaklines=true,columns=fullflexible}
\lstset{escapeinside={(*@}{@*)}}
\usepackage{fontspec}
\setmainfont{Arial}
\newfontfamily\stylishfont{Noteworthy}
%\newfontfamily\stylishfont{Zapfino}
\addbibresource{references.bib}
\usetikzlibrary{calc}
\usepackage{xcolor}

\lstdefinestyle{pidstyle}{
    basicstyle=\ttfamily\footnotesize,
    breaklines=true,
    escapechar=\#, % Define escape character for inline LaTeX commands
    linewidth=\textwidth,
    basicstyle=\ttfamily\scriptsize
}

\renewcommand{\maketitle}{%
  \begin{leftmark}
    \vspace*{\baselineskip} % Add a bit of vertical space

%    \includegraphics[width=4cm]{example-image-a} % Add an image before the title. you will need to replace the image path with your own

%    \vspace{0.5cm} % Add vertical space before title

    \textbf{\fontsize{18}{36}\selectfont \thetitle} % Font Size and Bold Title

     \vspace{0.05cm} % Add vertical space before subtitle
%    \textit{\Large \theauthor}  % Subtitle / Author
    \vspace{\baselineskip} % Add vertical space after subtitle
     \rule{\textwidth}{0.4pt} % Add a horizontal line

   \end{leftmark}
%    \thispagestyle{empty} % Prevent header/footer on the title page
}


% Section Formatting
\titleformat{\section}
  {\normalfont\fontsize{18}{22}\bfseries} % Font and style
  {\thesection}         % Section number
  {1em}                   % Horizontal space after section number
  {}                     % Code before the section name
  []                     % Code after the section name

\titleformat{\subsection}
  {\normalfont\fontsize{14}{18}\bfseries} % Font and style
  {\thesubsection}         % Subsection number
  {1em}                   % Horizontal space after subsection number
  {}                     % Code before the subsection name
  []                     % Code after the subsection name

\setlength{\parindent}{0pt}

\title{Computing platforms (Spring 2025)\newline
week 6}
\author{Juha-Pekka Heikkilä}



\pagestyle{fancy}
\fancyhf{}

\renewcommand{\headrulewidth}{0pt}

\newcommand{\footerline}{\makebox[\textwidth]{\hrulefill}}

\newcommand{\footercontent}{%
    \begin{tabular}{@{}l@{}}
        \footerline \\
        \leftmark \hfill \rlap{\thepage}
    \end{tabular}
}

\fancyfoot[C]{\footercontent}


\newcommand{\exercise}[1]{
    \section*{Exercise #1}
    \markboth{Exercise #1}{}
}


\begin{document}
\maketitle


\exercise{1}
\paragraph{Readers/Writers with priorities.} The lecture slides
give an example of solving the Readers/Writers problem with locks
and condition variables. That solution is inefficient in the sense
that when both writers and readers are waiting when a writer finishes,
they will fight over who will go next. The python code
“readers-writers.py” gives another solution that solves the problem.
Download the code separately from Moodle.

\begin{enumerate}[label=\textbf{\makebox[1cm][l]{\Huge\text{(\stylishfont\alph*)}}}, leftmargin=!, labelindent=0pt]
    \item Can a writer starve in this solution? Can a reader
    starve in this solution?
    \paragraph{Writer can starve.} Once writer finishes all readers are allowed to proceed
    in concurrent manner. This race can preempt waiting writer and
    possibly starve writer.
    \paragraph{Reader cannot starve.} When writer finish all readers can
    proceed simultaneously.

    \item Who has higher priority, readers or writers? Why?
    \paragraph{Readers have higher priority} When writer finishes all
    readers proceed together, this give advantage to readers.
    
    \item Modify the code so that writers have priority.
    \begin{lstlisting}
import threading

# Keep track of currently reading threads
readers=set()
# Keep track of currently writing threads
writers=set()
# Keep track of the number of readers waiting to read
waitingreaders=0
# Keep track of the number of writers waiting to write
waitingwriters=0
# Lock for protecting the shared variables
lock = threading.Lock()
# Condition variable where readers wait for their turn to read
oktoread = threading.Condition(lock)
# Condition variable where writers wait for their turn to write
oktowrite = threading.Condition(lock)

# Function executed by the readers
def reader():
    global readers
    global writers
    global waitingreaders
    global waitingwriters
    # Retrieve the name of this thread
    name = threading.current_thread().name
    # Each reader will read three times
    for _ in range(3):
        lock.acquire()
        (*@\colorbox{yellow}{\# If there are currently writers readers needs to wait}@*)
        while len(writers) > 0 (*@\colorbox{yellow}{or waitingwriters > 0}@*):
            waitingreaders += 1
            oktoread.wait()
            waitingreaders -= 1
        (*@\colorbox{yellow}{\# Now there are no writers}@*)
        readers.add(name)
        print(name + ": Starts reading")
        print("Current readers: " + str(readers) + " Current writers: " + str(writers))
        lock.release()
        print(name + ": Reading...")
        lock.acquire()
        print(name + ": Finishes reading")
        readers.remove(name)
        # If there are no readers left, let the writers start writing
        if len(readers) == 0:
            oktowrite.notify()
        lock.release()

# Function executed by the writers
def writer():
    global readers
    global writers
    global waitingreaders
    global waitingwriters
    # Retrieve the name of this thread
    name = threading.current_thread().name
    # Each writer writes three times
    for _ in range(3):
        lock.acquire()
        (*@\colorbox{yellow}{\# Modified for writer priority}@*)
        while len(writers) > 0 (*@\colorbox{yellow}{or len(readers) > 0}@*):
            waitingwriters += 1
            oktowrite.wait()
            waitingwriters -= 1
        # Now there are no current readers/writers
        writers.add(name)
        print(name + ": Starts writing")
        print("Current readers: " + str(readers) + " Current writers: " + str(writers))
        lock.release()
        print(name + ": Writing...")    
        lock.acquire()
        print(name + ": Finishes writing")
        writers.remove(name)
        (*@\colorbox{yellow}{\# Modified notification order: notify waiting writers first, then readers}@*)
        if waitingwriters > 0 :
            oktowrite.notify()
        elif waitingreaders > 0:
            oktoread.notify_all()
        lock.release()

# Create the reader and writer threads
readerThreads = [threading.Thread(target=reader, name = "R" + str(i)) for i in range(10)]
writerThreads = [threading.Thread(target=writer, name = "W" + str(i)) for i in range(10)]

# Start the reader and writer threads
for i in range(10):
    readerThreads[i].start()
    writerThreads[i].start()
        \end{lstlisting}
    
    \item Can a reader starve in your solution? Can a writer starve
    in your solution?
    \paragraph{Reader: yes, can starve.} if writers keep on arriving
    continously readers never get to execute.
    \paragraph{Writer: no, can\'t starve.} Writers have priority, if
    there is writer waiting readers will be blocked.

\end{enumerate}







\newpage
\exercise{2}
\paragraph{Deadlock.} (Problem 6.6 from [Stallings 2018], modified to Python) In the code
below, three threads are competing for six resources labeled A to F.
\begin{verbatim}
#thread 1       #thread 2       #thread 3
while True:     while True:     while True:
    get(A)          get(D)          get(C)
    get(B)          get(E)          get(F)
    get(C)          get(B)          get(D)
    # critical region: # critical region: # critical region:
    # use A, B, C   # use D, E, B   # use C, F, D
    release(A)      release(D)      release(C)
    release(B)      release(E)      release(F)
    release(C)      release(B)      release(D)
\end{verbatim}

\begin{enumerate}[label=\textbf{\makebox[1cm][l]{\Huge\text{(\stylishfont\alph*)}}}, leftmargin=!, labelindent=0pt]
    \item Using a resource allocation graph (Figures 6.5 and 6.6 in [Stallings 2018],
    also in lecture slides), show the possibility of deadlock in this implementation.\newline

    we have resource allocation graph with two types of vertices:
\begin{itemize}
    \item \textbf{Threads (T1, T2, T3)}.
    \item \textbf{Resources (A, B, C, D, E, F)}.
\end{itemize}
then we draw:
\begin{itemize}
    \item \(\text{T} \rightarrow \text{R}\) if thread T is requesting resource R.
    \item \(\text{R} \rightarrow \text{T}\) if resource R is currently allocated to thread T.
\end{itemize}

\paragraph{Possible interleaving that leads to deadlock:}
\begin{enumerate}
    \item \textbf{T1} has successfully acquired \(\{A, B\}\) and is requesting \(C\).
    \item \textbf{T2} has successfully acquired \(\{D, E\}\) and is requesting \(B\).
    \item \textbf{T3} has successfully acquired \(\{C, F\}\) and is requesting \(D\).
\end{enumerate}
From here we have:
\[
T1 \rightarrow C, \quad
C \rightarrow T3, \quad
T3 \rightarrow D, \quad
D \rightarrow T2, \quad
T2 \rightarrow B, \quad
B \rightarrow T1.
\]
E.g. a cycle:
\[
T1 \rightarrow C \rightarrow T3 \rightarrow D \rightarrow T2 \rightarrow B \rightarrow T1,
\]
Deadlock situation where each thread is holding resource next thread want.

\newpage
    \item Modify the order of some of the get requests to prevent the possiblity of any deadlock.
    You cannot move requests across threads, only change the order inside each thread. Use a
    resource allocation graph to justify your answer.\newline

    We can enforce \textbf{global ordering} of resource acquisition. Suppose we impose
    alphabetical order \(A < B < C < D < E < F\). Threads must request its resources
    in ascending order and release them in reverse order. A possible reordering is:

    \begin{itemize}
        \item \textbf{Thread 1}: \(\texttt{get(A)}\), \(\texttt{get(B)}\), \(\texttt{get(C)}\) 
        \item \textbf{Thread 2}: \(\texttt{get(B)}\), \(\texttt{get(D)}\), \(\texttt{get(E)}\)
        \item \textbf{Thread 3}: \(\texttt{get(C)}\), \(\texttt{get(D)}\), \(\texttt{get(F)}\)
    \end{itemize}
    
    No circular wait can form because there is no way to construct a cycle if every thread
    requests resources according to the same global ordering.
    
    \paragraph{Revised Resource Allocation Graph}
    \begin{itemize}
        \item \textbf{T1} edges always proceed \(T1 \rightarrow A \rightarrow B \rightarrow C\).
        \item \textbf{T2} edges always proceed \(T2 \rightarrow B \rightarrow D \rightarrow E\).
        \item \textbf{T3} edges always proceed \(T3 \rightarrow C \rightarrow D \rightarrow F\).
    \end{itemize}
    No thread requests resource with lower label than any resource it already holds hence we
    avoid circular-wait condition.
\end{enumerate}






\newpage
\exercise{3}

\paragraph{SSDs.} In this exercise, we use a Python simulator from OSTEP. You find
the code at https://github.com/remzi-arpacidusseau/ostep-homework/blob/master/file-ssd.
Download the Python code ssd.py and read the README of the simulator. Answer the homework
questions 1-5 from OSTEP Chapter 44. You find the homework questions at the end of the chapter
at https://pages.cs.wisc.edu/~remzi/OSTEP/file-ssd.pdf. For question 1 running with seed 
1 suffices.\newline

    1. The homework will mostly focus on the log-structured SSD, which
    is simulated with the “-T log” flag. We’ll use the other types of
    SSDs for comparison. First, run with flags -T log -s 1 -n 10
    -q. Can you figure out which operations took place? Use -c to
    check your answers (or just use -C instead of -q -c). Use different
    values of -s to generate different random workloads.
\scriptsize {
    \begin{verbatim}
        write(12, u)
        write(32, M)
        read(32) -> M
        write(38, 0)
        write(36, e)
        trim(36)
        read(32) -> M
        trim(32)
        read(12) -> u
        read(12) -> u
    \end{verbatim}
}
    2. Now just show the commands and see if you can figure out the
    intermediate states of the Flash. Run with flags -T log -s 2 -n
    10 -C to show each command. Now, determine the state of the
    Flash between each command; use -F to show the states and see if
    you were right. Use different random seeds to test your burgeoning
    expertise.
    \scriptsize {
        \begin{verbatim}
                                          State iiiiiiiiii iiiiiiiiii iiiiiiiiii iiiiiiiiii iiiiiiiiii iiiiiiiiii iiiiiiiiii
        cmd   0:: write(36, F) -> success State vEEEEEEEEE iiiiiiiiii iiiiiiiiii iiiiiiiiii iiiiiiiiii iiiiiiiiii iiiiiiiiii
        cmd   1:: write(29, 9) -> success State vvEEEEEEEE iiiiiiiiii iiiiiiiiii iiiiiiiiii iiiiiiiiii iiiiiiiiii iiiiiiiiii
        cmd   2:: write(19, I) -> success State vvvEEEEEEE iiiiiiiiii iiiiiiiiii iiiiiiiiii iiiiiiiiii iiiiiiiiii iiiiiiiiii
        cmd   3:: trim(19) -> success     State vvvEEEEEEE iiiiiiiiii iiiiiiiiii iiiiiiiiii iiiiiiiiii iiiiiiiiii iiiiiiiiii
        cmd   4:: write(22, g) -> success State vvvvEEEEEE iiiiiiiiii iiiiiiiiii iiiiiiiiii iiiiiiiiii iiiiiiiiii iiiiiiiiii
        cmd   5:: read(29) -> 9           State vvvvEEEEEE iiiiiiiiii iiiiiiiiii iiiiiiiiii iiiiiiiiii iiiiiiiiii iiiiiiiiii
        cmd   6:: read(22) -> g           State vvvvEEEEEE iiiiiiiiii iiiiiiiiii iiiiiiiiii iiiiiiiiii iiiiiiiiii iiiiiiiiii
        cmd   7:: write(28, e) -> success State vvvvvEEEEE iiiiiiiiii iiiiiiiiii iiiiiiiiii iiiiiiiiii iiiiiiiiii iiiiiiiiii
        cmd   8:: read(36) -> F           State vvvvvEEEEE iiiiiiiiii iiiiiiiiii iiiiiiiiii iiiiiiiiii iiiiiiiiii iiiiiiiiii
        cmd   9:: write(49, F) -> success State vvvvvvEEEE iiiiiiiiii iiiiiiiiii iiiiiiiiii iiiiiiiiii iiiiiiiiii iiiiiiiiii
        \end{verbatim}
    }
    3. Let’s make this problem ever so slightly more interesting by adding
    the -r 20 flag. What differences does this cause in the commands?
    Use -c again to check your answers.
    \scriptsize {
        \begin{verbatim}
        cmd   0:: read(41) -> fail: uninitialized read
        cmd   1:: write(33, j) -> success
        cmd   2:: write(30, A) -> success
        cmd   3:: read(33) -> j
        cmd   4:: write(49, W) -> success
        cmd   5:: write(22, g) -> success
        cmd   6:: read(23) -> fail: uninitialized read
        cmd   7:: read(22) -> g
        cmd   8:: write(28, e) -> success
        cmd   9:: read(33) -> j
        \end{verbatim}
    }
    We see failed commands.\newline

\newpage
    4. Performance is determined by the number of erases, programs, and
    reads (we assume here that trims are free). Run the same workload
    again as above, but without showing any intermediate states (e.g.,
    -T log -s 1 -n 10). Can you estimate how long this workload
    will take to complete? (default erase time is 1000 microseconds,
    program time is 40, and read time is 10) Use the -S flag to check
    your answer. You can also change the erase, program, and read
    times with the -E, -W, -R flags.\newline
    \scriptsize {
        \begin{verbatim}
            FTL   (empty)
            Block 0          1          2          3          4          5          6          
            Page  0000000000 1111111111 2222222222 3333333333 4444444444 5555555555 6666666666 
                  0123456789 0123456789 0123456789 0123456789 0123456789 0123456789 0123456789 
            State iiiiiiiiii iiiiiiiiii iiiiiiiiii iiiiiiiiii iiiiiiiiii iiiiiiiiii iiiiiiiiii 
            Data                                                                               
            Live                                                                               
            
            
            FTL    12:  0  38:  2 
            Block 0          1          2          3          4          5          6          
            Page  0000000000 1111111111 2222222222 3333333333 4444444444 5555555555 6666666666 
                  0123456789 0123456789 0123456789 0123456789 0123456789 0123456789 0123456789 
            State vvvvEEEEEE iiiiiiiiii iiiiiiiiii iiiiiiiiii iiiiiiiiii iiiiiiiiii iiiiiiiiii 
            Data  uM0e                                                                         
            Live  + +                                                                          
            
            Physical Operations Per Block
            Erases   1          0          0          0          0          0          0          Sum: 1
            Writes   4          0          0          0          0          0          0          Sum: 4
            Reads    4          0          0          0          0          0          0          Sum: 4
            
            Logical Operation Sums
              Write count 4 (0 failed)
              Read count  4 (0 failed)
              Trim count  2 (0 failed)
            
            Times
              Erase time 1000.00
              Write time 160.00
              Read time  40.00
              Total time 1200.00
        \end{verbatim}
    }

    5. Now, compare performance of the log-structured approach and the
    (very bad) direct approach (-T direct instead of -T log). First,
    estimate how you think the direct approach will perform, then check
    your answer with the -S flag. In general, how much better will the
    log-structured approach perform than the direct one?\newline

    Logical mapped:
    \scriptsize {
        \begin{verbatim}
            Times
            Erase time 1000.00
            Write time 160.00
            Read time  40.00
            Total time 1200.00
        \end{verbatim}
    }
    Direct mapped:
    \scriptsize {
        \begin{verbatim}
            Times
            Erase time 4000.00
            Write time 280.00
            Read time  70.00
            Total time 4350.00
        \end{verbatim}
    }




\newpage
\exercise{4}
\paragraph{UNIX file management.} Read Chapter 12.8 [Stalling, 2018, pp. 580-585] and
answer the following questions. Consider the organization of UNIX file as represented
by the inode (see Figure 12.15, also provided below). Assume there are 12 direct block
pointers, and a singly, doubly, and triply indirect pointer to each inode. Assume further
that the system block size and the disk sector size are both 8 KB, and the disk block
pointer is 32 bits, with 8 bits to identify the physical disk and 24 bits to identify
the physical block.

\begin{enumerate}[label=\textbf{\makebox[1cm][l]{\Huge\text{(\stylishfont\alph*)}}}, leftmargin=!, labelindent=0pt]
    \item What is the maximum file size supported by this system?
    
    \begin{itemize}
        \item Direct: 12 blocks * 8192B = 98304 B
        \item Singly indirect: 2048 pointers * 8192B = 16777216 B (16GB)
        \item Doubly indirect: \(2048^2\) * 8192 = 4194304 * 8192 \(\approx\) 34359738368 B (34GB)
        \item Triply indirect: \(2048^3\) * 8192 = 8589934592 * 8192 \(\approx\) 70368744177664 B (70TB)
    \end{itemize}
    
    \item What is the maximum system partition (i.e. physical disk) supported by this
    system?
    \begin{itemize}
        \item With 24 bits for physical block number:
          \[
          2^{24} = 16777216 \text{ blocks per disk}
          \]
        \item Each block is 8192 B:
          \[
          16777216 \times 8192 = 137438953472 \text{ B} = 2^{24+13} = 2^{37} \text{ B}
          \]
        \item maximum partition size:
          \[
          2^{37} \text{ B} \approx 137.44 \text{ GB}
          \]
      \end{itemize}

    \item Assuming no information other than the the file inode is already on
    the main memory, how many disk accesses are required to access the byte in
    position 13 423 956.

    \begin{itemize}[leftmargin=2em]
        \item block index:  
          \[
          \text{Block index} = \left\lfloor \frac{13\,423\,956}{8192} \right\rfloor \approx 1638
          \]
        \item Mapping:
          \begin{itemize}
            \item Direct pointers cover blocks 0 -11
            \item The singly indirect block covers blocks 12 to 12 + 2048 - 1 = 2059
          \end{itemize}
          1638 is in singly indirect range:
        \item Disk accesses needed (inode is already in memory):
          \begin{itemize}
            \item 1 access to read the singly indirect block.
            \item 1 access to read the actual data block.
          \end{itemize}
      \end{itemize}
\end{enumerate}





\newpage
\exercise{5}

\paragraph{Bob wants to analyze the network traffic on his computer.} To gather
the data, he creates a system that inspects every network packet that goes through
his network device. To store the data, he creates a new file for each network packet
using a timestamp as a file name and stores the size of the network packet in the file.
Soon he has gathered huge amounts of data and he has a huge amount of small files in
his system.

\begin{enumerate}[label=\textbf{\makebox[1cm][l]{\Huge\text{(\stylishfont\alph*)}}}, leftmargin=!, labelindent=0pt]
    \item What are the disadvantages of this approach?
      \begin{itemize}
        \item Excessive file system overhead.
        \item Poor performance due to many small file creations and system calls.
        \item Increased fragmentation and slower file system operations.
        \item Difficulty managing and backing up a huge number of files.
      \end{itemize}

    \item After running this system for some time, the whole computer system crashed.
    What could be the reason for this?
    \begin{itemize}
        \item Running out of storage space or inodes due to an enormous number of files.
        \item Overloaded file system metadata management causing memory exhaustion or
        degraded system performance.
      \end{itemize}
    \item How should the data be organized instead?
      \begin{itemize}
        \item Store data into larger files. e.g., appending many packets  to a single file.
        \item Organize data by connection or time intervals rather than per packet.
        \item maybe use database that can efficiently store and index numerous records.
      \end{itemize}
\end{enumerate}

\newpage
\printbibliography
\end{document}
