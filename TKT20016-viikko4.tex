\documentclass{article}
\usepackage[utf8]{inputenc}
\usepackage{graphicx}
\usepackage{titling}
\usepackage{titlesec}
\usepackage{booktabs}
\usepackage{fancyhdr}
\usepackage{lipsum}
\usepackage{comment}
\usepackage{enumitem}
\usepackage{listings}
\usepackage{xcolor}
\usepackage{longtable}
\usepackage{cite}
\usepackage{pgfgantt}
\usepackage{amsmath}
\usepackage{tikz}
\usepackage[margin=1in]{geometry}
\usetikzlibrary{calc}


\lstdefinestyle{pidstyle}{
    basicstyle=\ttfamily\footnotesize,
    breaklines=true,
    escapechar=\#, % Define escape character for inline LaTeX commands
    linewidth=\textwidth,
    basicstyle=\ttfamily\scriptsize
}

\renewcommand{\maketitle}{%
  \begin{leftmark}
    \vspace*{\baselineskip} % Add a bit of vertical space

%    \includegraphics[width=4cm]{example-image-a} % Add an image before the title. you will need to replace the image path with your own

%    \vspace{0.5cm} % Add vertical space before title

    \textbf{\fontsize{18}{36}\selectfont \thetitle} % Font Size and Bold Title

     \vspace{0.05cm} % Add vertical space before subtitle
%    \textit{\Large \theauthor}  % Subtitle / Author
    \vspace{\baselineskip} % Add vertical space after subtitle
     \rule{\textwidth}{0.4pt} % Add a horizontal line

   \end{leftmark}
%    \thispagestyle{empty} % Prevent header/footer on the title page
}


% Section Formatting
\titleformat{\section}
  {\normalfont\fontsize{18}{22}\bfseries} % Font and style
  {\thesection}         % Section number
  {1em}                   % Horizontal space after section number
  {}                     % Code before the section name
  []                     % Code after the section name

\titleformat{\subsection}
  {\normalfont\fontsize{14}{18}\bfseries} % Font and style
  {\thesubsection}         % Subsection number
  {1em}                   % Horizontal space after subsection number
  {}                     % Code before the subsection name
  []                     % Code after the subsection name

\setlength{\parindent}{0pt}

\title{Computing platforms (Spring 2025)\newline
week 4}
\author{Juha-Pekka Heikkilä}



\pagestyle{fancy}
\fancyhf{}

\renewcommand{\headrulewidth}{0pt}

\newcommand{\footerline}{\makebox[\textwidth]{\hrulefill}}

\newcommand{\footercontent}{%
    \begin{tabular}{@{}l@{}}
        \footerline \\
        \leftmark \hfill \rlap{\thepage}
    \end{tabular}
}

\fancyfoot[C]{\footercontent}


\newcommand{\exercise}[1]{
    \section*{Exercise #1}
    \markboth{Exercise #1}{}
}



\begin{document}
\maketitle


\exercise{1}
Search the websites of \textbf{AWS, Google Cloud,} and 
\textbf{Microsoft Azure} and find out answers to the following
questions for each 3 providers:

\begin{enumerate}
  \item What services can you use for free? \textbf{All of them }
  \begin{itemize}
    \item https://aws.amazon.com/free \cite{aws_free}\newline
      Free for 12 months.
    \item https://cloud.google.com/free \cite{gcp_free} \newline
      Always free.
    \item https://azure.microsoft.com/en-us/free/ \cite{azure_free} \newline
      Free for 12 months.  
  \end{itemize}


  \item How much does it cost to run a virtual machine for
  one day? (Something other than a possibly free plan.)
  \begin{itemize}
    \item AWS: t4g.xlarge (4 vCPUs, 16 GB RAM) costs \$0.1344/hour, \$3.2256/day.
    \item Google Cloud: e2-standard-4 (4 vCPUs, 16 GB RAM) costs \$0.150924/hour, totaling \$3.622176/day.
    \item Microsoft Azure: B4ms (4 vCPUs, 16 GB RAM) costs \$0.166/hour, totaling \$3.984/day.
  \end{itemize}

  \item How much does storage cost?
  \begin{itemize}
    \item AWS: \$0.023 per GB per month.
    \item Google Cloud: \$0.023 per GB per month.
    \item Microsoft Azure: \$0.021 per GB per month.
  \end{itemize}

  \item Compare prices and expected performance
  of different VM offerings. Does the price increase
  linearly with the expected performance? If not, what can
  you observe? \newline
  The price does not increase linearly with performance
  \cite{cloud_pricing_comparison}. Factors such as instance
  type, underlying hardware, and specific optimizations
  influence pricing. Compute-optimized instances offer
  higher performance per vCPU at a lower cost compared
  to general-purpose instances.

  \item Which of the three is best?\newline
  The "best" provider depends on use case:
  \begin{itemize}
    \item AWS: Broad service range.
    \item Google Cloud: Strong in data analytics and machine learning.
    \item Azure: Best for Microsoft-integrated environments.
  \end{itemize}
  
  \item Suppose you work in a company that wants to start
  using cloud computing. Which of the three would you pick
  and why?
  The choice depends on company needs:
  \begin{itemize}
    \item For a Microsoft-centric environment, Azure
    is preferable.
    \item For scalable web applications, AWS offers
    a broad service range.
    \item For data-intensive applications, Google Cloud
    excels in analytics and AI.
  \end{itemize}

\end{enumerate}




\newpage

\exercise{2}
Would cloud computing have taken off in this
manner without virtualization (VMs or containers)?
Why or why not?
  
  Cloud computing as we know it today would not
  have taken off in the same manner without virtualization
  (VMs and containers). Here’s why:

  \begin{itemize}
    \item \textbf{Efficient Resource Utilization:} Without 
    virtualization, entire physical machines would have to be
    allocated to each customer, leading to significant resource
    waste. Virtualization enables multiple customers to share
    the same physical server, reducing costs and maximizing
    hardware utilization.

    \item \textbf{Scalability and Elasticity:} Virtualization
    allows for on-demand provisioning and scaling of resources.

    \item \textbf{Disaster Recovery and Redundancy:}
    Virtualization enables snapshotting, live migration,
    and failover mechanisms, which are critical for disaster
    recovery.

    \item \textbf{Isolation and Security:} VMs and containers
    provide isolation between workloads, ensuring that multiple
    users can run applications on the same physical server
    without interfering with each other.

    \item \textbf{Cost Reduction:} Cloud computing relies on
    the economies of scale made possible by virtualization.
    
    \item \textbf{Portability and Flexibility:} Containers,
    in particular, allow applications to be easily moved between
    environments (e.g., development, testing, and production).
    
    \item \textbf{Rapid Deployment and Automation:} With
    virtualization, cloud providers can automate deployment,
    orchestration, and management of services.

  \end{itemize}




  \newpage

\exercise{3}
Cloud computing. Read the article “A View of Cloud Computing” by M. Armbrust et al.
available at https://dl.acm.org/doi/10.1145/1721654.1721672. The article is written in 2010
and makes 10 “predictions” (obstacles and opportunities) about how cloud computing will develop.
Based on the article, your knowledge about cloud computing, and any other material you may find,
address each of the 10 points. For each of them, classify them as “mostly true”, “partially true”, or
“mostly untrue” and justify your answer. The classification should reflect the current state of that
point in modern cloud computing. Note that in many of the points, several options are justifiable and
no correct answers exist.

\begin{itemize}
  \item \textbf{Availability/Business Continuity:}\newline
  Mostly True\newline

  The concern in 2010 was whether cloud providers could offer
  high availability comparable to traditional IT infrastructure.
  Today, major cloud providers like AWS, Google Cloud,
  and Microsoft Azure have multi-region availability zones,
  automated failover mechanisms, and SLAs guaranteeing 99.9\%+ 
  uptime. While cloud outages still occur, they are generally
  less frequent than on-premise failures.

  \item \textbf{Data Lock-in:}
  Partially True\newline

  Cloud providers still use proprietary storage APIs and
  management tools, making it difficult for customers
  to migrate workloads seamlessly. However, the situation
  has improved with industry-standard APIs, multi-cloud
  strategies, and containerization technologies
  
  \item \textbf{Data Confidentiality and Auditability:}
  Partially True\newline

  Security in cloud computing has improved significantly
  with end-to-end encryption, zero-trust architectures,
  and compliance certifications. However, enterprises
  still express concerns about data sovereignty and
  government access to data, particularly in public cloud
  environments.
\end{itemize}




\newpage

\exercise{4}

\begin{itemize}
  \item \textbf{Data Transfer Bottlenecks:}\newline
  Partialy True\newline

  In 2010, network bandwidth limitations were a major
  concern for cloud computing adoption. Today, network
  speeds have improved significantly.However, moving
  large datasets (petabytes of data) between clouds
  or from on-premises environments is still slow and costly.

  \item \textbf{Performance Unpredictability:}\newline
  Mostly True\newline

  Cloud environments share physical resources among
  multiple tenants, leading to resource contention.
  This is especially true for network I/O, disk I/O,
  and CPU allocation on virtual machines. While cloud
  providers have introduced dedicated instances, custom
  VM shapes, and optimized scheduling mechanisms,
  performance variability still exists.

  \item \textbf{Scalable Storage:}\newline
  Mostly True\newline

  Cloud providers have largely solved the problem
  of scalable storage through object storage,
  distributed databases and auto-scaling block storage
  solutions. However, latency-sensitive applications
  (e.g., real-time processing) may still face challenges
  with data consistency and retrieval speeds.
\end{itemize}

\newpage




\exercise{5}

\begin{itemize}
  \item \textbf{Bugs in Large Distributed Systems:}\newline
  Mostly True\newline

  Debugging large-scale distributed systems remains
  a challenge, as cloud environments introduce complexity,
  race conditions, and distributed failures that are difficult
  to reproduce in test environments.

  \item \textbf{Scaling Quickly:}\newline
  Mostly True\newline

  One of the biggest successes of cloud computing is its
  ability to scale quickly.

  \item \textbf{Reputation Fate Sharing:}\newline
  Partially True\newline

  The concern that one cloud user's bad behavior could
  impact others remains valid but has been mitigated
  by network isolation, security policies, and managed
  services. IP blacklisting and shared infrastructure
  risks still exist, particularly in multi-tenant
  environments where one customer's activity can lead
  to service degradation or reputational damage

  \item \textbf{Software Licensing:}\newline
  Partially True\newline

  In 2010, traditional software licensing models
  did not align well with cloud computing.
  Since then, many software vendors have adopted
  subscription-based and pay-as-you-go pricing models.
  Still, some enterprise software still has restrictive
  licensing that complicates cloud migration.
\end{itemize}


\newpage
\bibliographystyle{plain}
\bibliography{references}
\end{document}